\onecolumn
%Checklist:
%%%%%%%%%%%%%%%% 
%% Checklist-specific precommands:
\newenvironment{checklist_font}{\fontfamily{lmss}\small\selectfont}{\par}
\newlist{todolist}{itemize}{2}
\setlist[todolist]{label=$\square$}
%% start checklist:
\begin{checklist_font}
\begin{itemize}
\begin{todolist}

%%%%%%%%%%%%%%%%%
%%%%%% step 0
\begin{tcolorbox}[width=15cm, center]
    {\large {\color{teal}STEP 0: KNOW WHAT YOU WANT TO SIMULATE}} \newline
    \textbf{Initial questions you should ask before you set up an alchemical free energy calculation using molecular dynamics simulations}
    \item Do I understand the biology, chemistry and physics of my system?
    \item Have I properly prepared my protein and ligand systems?
    \item Does my system contain any structures that require custom parameters?
    \item What simulation protocol will provide the most evidence to answer my hypothesis?
    \item Are the projected computational expense and runtime realistic for my goals?
    \item Will my protocol be reproducible? 
    \item Will my statistics be reliable? If not, would more replicates solve the problem? Three replicates is the minimal number of replicates that should be run
    \item Can I open-source my data?
\end{tcolorbox}

%%%%%% step 1
\begin{tcolorbox}[width=15cm, center]
    {\large {\color{teal}STEP 1: PREPARING YOUR SIMULATIONS}} \newline
    \textbf{Steps to getting started setting up your alchemical free energy calculation}
    \item Make sure you know why you have picked your (combination of) force field(s)
    \item Minimize your system
    \item Equilibrate your system with your choice of thermodynamic ensemble
    \item Check the stability of your system and whether it behaves the way you want it to
\end{tcolorbox}

%%%%%% step 2
\begin{tcbraster}[raster columns=2,raster equal height, raster width=18cm]
\begin{adjustwidth}{-4.3cm}{}           % workaround to make the dual box pass the margin
\begin{todolist}
    \begin{tcolorbox}
    % absolute
        {\large {\color{teal}STEP 2: RUNNING ABSOLUTE SIMULATIONS}} \newline
        \textbf{Steps to running your absolute alchemical free energy calculations}
        
        \item Your ligands should have the same, biologically correct binding pose
        \item Make sure your \textlambda-scheduling is set correctly
        \item Check if your ligands are discharging and vanishing correctly
        \item Set up your restraints correctly
        \item Make sure you subsample in your MBAR protocol
        \item Apply the right correction terms
    \end{tcolorbox}
    \begin{tcolorbox}
    % relative
        {\large {\color{teal}STEP 2: RUNNING RELATIVE SIMULATIONS}} \newline
        \textbf{Steps to running your relative alchemical free energy calculations}
        \item           % circumvents bug in tcbraster/todolist
        \item Your ligands should have the same, biologically correct binding pose
        \item Make sure your λ-scheduling is set correctly
        \item Make sure your molecular transformations are realistic (1-5 heavy atoms for reliable computations)
        \item Generate a perturbation network by your method of choice; check whether you have enough cycle closures (preferably, every ligand should be in a cycle)
        \item Check whether dummy atoms were assigned correctly
        \item Make sure you subsample in your MBAR protocol
    \end{tcolorbox}
    
\end{todolist}
\end{adjustwidth}
\end{tcbraster}


%%%%%% step 3
\begin{tcolorbox}[width=15cm, center]
    {\large {\color{teal}STEP 3: HOW DO I KNOW WHICH SIMULATIONS ARE NOT RELIABLE?}} \newline
    \textbf{Situations suggesting your relative alchemical free energy calculations have not run properly (assuming absence of experimental affinities)}
    \item Standard error (\textsigma) should be \textless1 kcal·mol-1 
    \item Simulated systems have not converged 
    \newline\newline\textit{Relative:}
    \item If you observe hysteresis in perturbations and incorrect cycle closures
    \item Energy differences \textgreater15 kcal·mol-1  are unreliable
    \newline\newline\textit{Absolute:}
    \item Energies below -15 kcal·mol-1  are unreliable
    \item The ligand has not sampled most of the intended region after the vanishing step
    \item The ligand is drifting out of the intended region after the vanishing step
\end{tcolorbox}

%%%%%% step 4
\begin{tcolorbox}[width=15cm, center]
    {\large {\color{teal}STEP 4: WHY ARE THEY NOT RELIABLE?}} \newline
    \textbf{Suggestions for finding out why your alchemical free energy calculations may not be reliable}
    \item Check again whether dummy atoms were assigned correctly
    \item Inspect the trajectories across the λ-schedule (particularly the endpoints)
    \item Inspect the overlap matrices
\end{tcolorbox}

%%%%%% step 
\begin{tcolorbox}[width=15cm, center]
    {\large {\color{teal}STEP 5: DATA ANALYSIS}} \newline
    \textbf{Steps to analyzing your output data correctly}
    \item Make sure you have run enough replicates to ensure statistical reliability (\textgreater3)
    \item Compute both correlation and ranking coefficients and ranking statistics (e.g. r, \textrho, MUE and \texttau)
    \item You should be able to explain large energy differences based on your system (trajectory)
\end{tcolorbox}


%%%%%%%%%%%%%%%% 
% end checklist:
\end{todolist}
\end{itemize}
\end{checklist_font}
%%%%%%%%%%%%%%%%%
\twocolumn